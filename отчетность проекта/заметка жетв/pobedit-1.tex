\documentclass{jetpl}
\twocolumn

%%% article in English
\lat

%%% declaration of a new mathematical operator
\DeclareMathOperator{\sign}{sign}

%%% article title
\title{Компьютерный анализ структуры срезов карбида вольфрама}

%%% article title - for colontitle (at the top of the page)
\rtitle{Компьютерный анализ структуры \ldots}

%%% article title - for table of contents (usualy identical with \title)
\sodtitle{Компьютерный анализ структуры срезов карбида вольфрама}

%%% author(s) ( + e-mail)
\author{Д.\,Г.\,Каграманян$^{+}$,Е.\,П.\,Константинова$^{+}$,
Б.\,Б.\,Страумал$^{*}$, Л.\,Н.\,Щур$^{+\#}$\/\thanks{e-mail: lev@landau.ac.ru}}

%%% author(s) - for colontitle (at the top of the page)
\rauthor{Каграманян Д.Г., Константинова Е.П., Страумал Б.Б., Щур Л.Н.}

%%% author(s) - for table of contents
\sodauthor{Каграманян, Константинова, Страумал, Щур}

%%% author's address(es)
\address{$^+$Национальный исследовательский университет Высшая школа экономики, 101000, Москва\\~\\
$^+$Институт физики твердого тела РАН, 142432, Черноголовка\\~\\
$^\#$Институт теоретической физики им. Л.Д. Ландау РАН, 142432, Черноголовка}

%%% dates of submition & resubmition (if submitted once, second argument is *)
\dates{1 Июня 2021}{12 июня 2021}

%%% abstract
\abstract{Предложен подход для анализа структуры карбида вольфрама с использованием компьютерной обработки фотографий срезов. Из цифровых фотографий, полученных электронным сканирующим микроскопом, извлекаются такие геометрические параметры кобальтовой связки. В частности, обнаружено, что распределение внутренних углов срезов кобальтовой связки имеет выраженную бимодальную структуру, причем значения углов пиков практически одинаковы для образцов с мелко-, средне- и крупно-дисперсными карбидовыми включениями. Получены также распределения полуосей кобальтовых связок. Мы излагаем  детали разработанных методов обработки изображений срезов. }

%%% PACS numbers
\PACS{74.50.+r, 74.80.Fp}

\begin{document}

\maketitle

Материалы на основе карбида вольфрама WC («победиты», cemented carbides) известны уже около ста лет. Исследования «победитов» привели к их широкому использованию в технологических процессах и в областях промышленности, например, это бурение многокилометровых скважин и сверление субмиллиметровых отверстий. Победиты применяются в горном деле, гражданском и дорожном строительстве, машиностроении и при проходке тоннелей. По-видимому, еще далеко не исчерпаны возможности для формирования и улучшения их структуры, для чего проводятся разработки новых подходов. В статье мы предлагаем один из таких подходов с использованием элементов искусственного интеллекта.
	Сам по себе карбид вольфрама обладает очень высокой твердостью, близкой к максимально возможной. Однако, он весьма хрупок. Поэтому первый работоспособный материал с его использованием был получен за счет погружения зерен карбида вольфрама в пластичную металлическую связку на основе кобальта Co [1]. Варьируя долю карбида вольфрама и кобальтовой связки в материале, а также состав кобальтовой связки, размер зерен карбида вольфрама и другие параметры, удалось добиться широкого спектра и сочетания свойств, необходимых для различных применений, таких как твердость, износостойкость, трещиностойкость, пластичность и ударная вязкость [2]. Для дальнейшего развития этого класса материалов недостаточно обычных параметров описания микроструктуры, которые используются в количественной металлографии, таких как средний размер зерен или объёмная доля фаз. Необходим поиск тонких инструментов, позволяющих охарактеризовать особенности топологии взаимного проникновения фаз WC и Co. В настоящей работе мы приводим результаты первого шага на этом пути.

.

1. Описание трех образцов. Рисунки - три фото образцов (с мелко-, средне- и крупно-дисперсными карбидовыми включениями).
Описание метода предобработки изображений.

2. Описание метода определения границы.  Рисунок - Пример границы кобальтовых связок для одного из образцов.

3. Описание метода определения углов.  Рисунок - распределение углов для трех образцов.

4. Описание метода определения полуосей. Рисунок - распределение полуосей для трех образцов.

5. Обсуждение результатов.

Благодарности


\begin{thebibliography}{99}
\bibitem{kulik}
I.\,O. Kulik and A.\,N. Omel'yanchuk, Fiz.~Nizk.~Temp. {\bf 3}, 945
(1977) [translated in Sov. J.~Low~Temp.~Phys.].

\bibitem{been1}
C.\,W. Beenakker and H. van Houten, Phys.~Rev.~Lett. {\bf 66}, 3056
(1991).

\bibitem{shumeiko1}
V.\,S. Shumeiko, E.\,N. Bratus', G. Wendin, Fiz.~Nizk.~Temp. {\bf 23},
249 (1997) [Sov.~J.~Low Temp.~Phys. {\bf 23}, 181 (1997)],
cond\D mat/9610101.

\bibitem{been2}
C.\,W. Beenakker, Phys.~Rev.~Lett. {\bf 67}, 3836 (1991).

\bibitem{if2}
D.\,A. Ivanov and M.\,V. Feigel'man, cond\D mat/9808029.

\end{thebibliography}
\end{document}
