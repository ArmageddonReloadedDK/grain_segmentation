%%%%%%%%%%%%%%%%%%%MAIN_OPTIONS%%%%%%%%%%%%%%%%%%%
\documentclass[a4paper, 14pt]{article}

%% Работа с русским языком
\usepackage{cmap}					% поиск в PDF
\usepackage{hyperref}				% гиперссылки
\usepackage[warn]{mathtext} 		% русские буквы в формулах
\usepackage[T2A]{fontenc}			% кодировка
\usepackage[utf8]{inputenc}			% кодировка исходного текста
\usepackage[english,russian]{babel}	% локализация и переносы

%% Дополнительная работа с математикой
\usepackage{amsfonts,amssymb,amsthm,mathtools} % AMS
\usepackage{amsmath}
\usepackage{icomma} % "Умная" запятая: $0,2$ --- число, $0, 2$ --- перечисление

%% Номера формул
%\mathtoolsset{showonlyrefs=true} % Показывать номера только у тех формул, на которые есть \eqref{} в тексте.

%%FONTS_Packadges
\usepackage{euscript} % Шрифт Евклид
\usepackage{mathrsfs} % Красивый матшрифт

%% Свои команды
\DeclareMathOperator{\sgn}{\mathop{sgn}}

%% Перенос знаков в формулах (по Львовскому)
\newcommand*{\hm}[1]{#1\nobreak\discretionary{}
	{\hbox{$\mathsurround=0pt #1$}}{}}

%%% Работа с картинками
\usepackage{graphicx}  % Для вставки рисунков
\graphicspath{{pictures/}{images2/}}  % папки с картинками
\setlength\fboxsep{3pt} % Отступ рамки \fbox{} от рисунка
\setlength\fboxrule{1pt} % Толщина линий рамки \fbox{}
\usepackage{wrapfig} % Обтекание рисунков и таблиц текстом
\usepackage[section]{placeins}
\usepackage{subcaption}

%% Работа с таблицами
\usepackage{array,tabularx,tabulary,booktabs} % Дополнительная работа с таблицами
\usepackage{longtable}  % Длинные таблицы
\usepackage{multirow} % Слияние строк в таблице

%%Links
\hypersetup{
	colorlinks=true,
	linkcolor=black,
	filecolor=magenta,      
	urlcolor=blue,
}

%%% Программирование
\usepackage{etoolbox} % логические операторы

%%% Страница
\usepackage{extsizes} % Возможность сделать 14-й шрифт
\usepackage{geometry} % Простой способ задавать поля
\geometry{top=25mm}
\geometry{bottom=35mm}
\geometry{left=20mm}
\geometry{right=20mm}
\usepackage{indentfirst}
%
\usepackage{fancyhdr} % Колонтитулы
\pagestyle{fancy}
\renewcommand{\headrulewidth}{0mm}  % Толщина линейки, отчеркивающей верхний колонтитул
%\lfoot{Нижний левый}
%\rfoot{Нижний правый}
%\rhead{Верхний правый}
%\chead{Верхний в центре}
%\lhead{Верхний левый}
% \cfoot{Нижний в центре} % По умолчанию здесь номер страницы

\usepackage{setspace} % Интерлиньяж
%\onehalfspacing % Интерлиньяж 1.5
%\doublespacing % Интерлиньяж 2
%\singlespacing % Интерлиньяж 1

\usepackage{multicol,caption}

\newenvironment{Figure}
{\par\medskip\noindent\minipage{\linewidth}}
{\endminipage\par\medskip}

\usepackage{enumitem}
\usepackage{amssymb}
\usepackage{xcolor}
%%% Зачеркнутый текст
\usepackage[normalem]{ulem}


\author{Каграманян Давид}
\title{Материал встречи}
\date{\today}

\begin{document}
	\thispagestyle{empty}
	\hfill\begin{minipage}{0.4\textwidth}
	    Каграманян Давид Геворгович БИВ184\\
	    dgkagramanyan@miem.hse.ru\\
	    Проект 398\\

	\end{minipage}%

	\begin{center}


		\textbf{\textit{Победит}}
		
		\vspace{1ex}
		
		\textbf{Итог встречи 29.12.2020}
	

	\end{center}

	\section{Объeкт исследований}
	
	В проекте рассматривается связка WC-Co с различным процентым соотношением карбида вольфрама и кобальта.
	Она получена путем жидкофазного спекания.
	 
	Перед началом исследований срез сначала шлифуется и полируется с помощью наждачной бумаги и алмазной
	полировальной пасты. Средняя высота бугров поверхности меньше 1 \textmu м, или 1000 \r A. 
	
	\section{Где используется}
	
	Основное применение - покрытие бурильных головок, сверел и всего того, что предназначено бурить твердые материалы.
	 
	\section{Оборудование и фотографии}
	 
	 Сканирующий микроскоп VegaTescan. Цифры, входящие в состав названия файла, указывают на процентные соотношения карбида вольфрама и кобальта (уточнить как именно).

	На каждой фотографии присутствует линейка. Ее длина для данного микроскопа  50 \textmu м.
	Изображение слева получено на основе отраженных электронов, а справа - на поглощенных. 
	Зерно кобальта увидеть нельзя. Количество частичек карбида прмерно 300 на одной фотографии. 
	
		\newpage
	\section{Анализ сплава и полученных фотографий}
	
	В проекте можно рассмотреть такие физические характеристики сплава, как:
	
	\begin{itemize}
		\item микротвердость(по Бренелю)
		
		\item ударная вязкость
		
		\item кривая наноиндентирования
		
		\item износостойкость
	\end{itemize}


	Для использования в качестве входных данных нейросети можно использовать следующие
	характеристики частиц карбида, полученные при обработке фотографий:
	
	\begin{itemize}
		\item количество соседей 
		
		\item количество и типы дыр между частицами
		
		\item несоосность форм
		
		\item типы границ
		
		\item контактные углы
		
		\item распределение по формам (площадь, периметр, углы и тд) 
		
		\item связность (?)
	
		\item расположение 
	\end{itemize}
	
	\section{Методы анализа фотографий}
	
	Для определения форм, распределений и прочих характеристик можно использовать готовые 
	инструменты из области Computer Vision (CV). 
	Также стоит рассмотреть количественную металлографию. 
	

	


	
\end{document}